\documentclass[12pt,a4paper,oneside]{article}
\usepackage{geometry}
\geometry{left=3cm, right=3cm, top=3.5cm, bottom=3.5cm}
\pagestyle{plain}
\usepackage[utf8]{inputenc}
\usepackage{lmodern}
\usepackage{float}
\usepackage{amsmath,amscd,amsfonts,amssymb,amstext,amsthm}
\usepackage{tikz}
\usetikzlibrary{matrix,arrows,patterns}
\usepackage{enumitem}
\usepackage{setspace,sectsty}

\title{Grading: Problem 2}
\date{\vspace{-6ex}\today}

\begin{document}
\maketitle

For $N=120$, $r=0.02$, $m=12$, calculate the future value to be paid $A$ amount at the beginning of the each year:

\begin{equation}
  FV = \sum_{i=1}^{N}\left(1+\frac{r}{m}\right)^i =
  \left(1+\frac{r}{m}\right)
  \frac{\left(1+\frac{r}{m}\right)^{N}-1}{\frac{r}{m}}=
  x\frac{x^N-1}{x-1},
\end{equation}
for $x=1+r/m$ and $FV$ is divided by $A$.
This is the money which is saved during $120$ months. Now, $1500\$ $ will be withdrawn at the beggining of the each following $360$ months. For $M=360$, the present value can computed as
\begin{equation}
  PV = \sum_{i=0}^{M-1}1500\left(1+\frac{r}{m}\right)^{-i} =
  1500\frac{1-\left(1+\frac{r}{m}\right)^{-M}}
  {1-\left(\frac{1}{1+\frac{r}{m}}\right)}
  = 1500 \frac{1-x^{-M}}{1-\frac{1}{x}}.
\end{equation}
for $x=1+r/m$.
Hence, the future and the present value must be equal to each other and $A$ can be computed as
\begin{equation}
  A = \frac{PV}{FV} = 3057.44
\end{equation}

\vspace{6ex}
\textbf{Explanation:} There is a mistake in the formulation of PV. (-0.5)

\textbf{Given point:} $3.5$

\end{document}